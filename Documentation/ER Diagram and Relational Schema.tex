\documentclass[12pt]{report}

\usepackage[utf8]{inputenc}
\usepackage{amsfonts}
\usepackage{graphicx}
\usepackage{amsmath}
\usepackage{geometry}

\graphicspath{..\ER Diagram}

\title{
    \textbf{Database Design for Online Retail Store\\ Winter 2022 \\}
}

\author{Abhimanyu Bhatnagar, 2020273
        \\Atyant Sony, 2020039
        \\Ritika Nagar, 2020112
        \\Madhava Krishna, 2020217}
\date{\textit{\today}}



\begin{document}
    \maketitle
    
    \tableofcontents
    \listoffigures
    \pagebreak

    %chapter 1, defining the problem
    \chapter{Defining the Problem}
    \section{Objective}
        The goal is to design a database management system for an online retail store,
        similar to Big Bazaar, Flipkart and Amazon. Our project models on Grofers (now BlinkIt), 
        a fast grocery marketplace for consumers to purchase day-to-day goods from.
    
    \section{Problem Statement}
    The ER diagrams and the relational schema that follow are based on the following (rudimentary) problem statement:
    \\\\
    %Problem statement
    \textit{
        The online retail store serves many customers. 
        The customers are required to hold an account on the platofrm to be able to purchase items.
        They can create an account by specifying their name, email address, phone number, and address.
        Customers add products to a shopping cart. They apply coupons on the shopping cart; the coupons 
        have a coupon code and an associated discount percentage.\\
        Customers order items by checking out the items on their shopping cart. The order is placed once the
         transaction is confirmed.
        \\
        A product can belong to various categories and has specifications and a cost. Each
        product is obtained from a vendor which the store transacts with. After being purchased from 
        the vendor, the products are stored in a warehouse. Warehouse employees are responsible for packing 
        and preparing orders. The readied order is then delivered to the customer by a delivery agent.
        \\
        In case of any lapses with an order, the customer complains to support staff who create a complaint
        number against the order. They send out the details regarding the complaint to the customer.
    }

    \chapter{ER Diagram}
    \includegraphics{'ER Diagram\ER Diagram'}

\end{document}